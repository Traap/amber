\paragraph{Test Case: t003}
\begin{description}[align=right,leftmargin=*,labelindent=3cm]
\item[Purpose:] Demonstrate requirement R789 is met.
\item[Requirement:] R789
\end{description}
\begin{lstlisting}[numbers=left]
       Step: 1
    Confirm: Program nice has installed.
Expectation: nice installation location is displayed.
    Command: which  nice 
Test Result: PASS
   Evidence: Starts on next line.
/usr/bin/nice

\end{lstlisting}
\begin{lstlisting}[numbers=left]
       Step: 2
    Confirm: Program nl has installed.
Expectation: nl installation location is displayed.
    Command: which  nl 
Test Result: PASS
   Evidence: Starts on next line.
/usr/bin/nl

\end{lstlisting}
\begin{lstlisting}[numbers=left]
       Step: 3
    Confirm: Program man has been installed.
Expectation: nroff installation location is displayed.
    Command: which  nroff 
Test Result: PASS
   Evidence: Starts on next line.
/usr/bin/nroff

\end{lstlisting}
\begin{lstlisting}[numbers=left]
       Step: 4
    Confirm: Program man has been installed.
Expectation: ed installation location is displayed.
    Command: which  ed 
Test Result: PASS
   Evidence: Starts on next line.
/usr/bin/ed

\end{lstlisting}
\begin{lstlisting}[numbers=left]
       Step: 5
    Confirm: A developer is able to access Git help.
Expectation: Git help is displayed.
    Command: git  help 
Test Result: PASS
   Evidence: Starts on next line.
usage: git [--version] [--help] [-C <path>] [-c <name>=<value>]
           [--exec-path[=<path>]] [--html-path] [--man-path] [--info-path]
           [-p | --paginate | --no-pager] [--no-replace-objects] [--bare]
           [--git-dir=<path>] [--work-tree=<path>] [--namespace=<name>]
           <command> [<args>]

These are common Git commands used in various situations:

start a working area (see also: git help tutorial)
   clone      Clone a repository into a new directory
   init       Create an empty Git repository or reinitialize an existing one

work on the current change (see also: git help everyday)
   add        Add file contents to the index
   mv         Move or rename a file, a directory, or a symlink
   reset      Reset current HEAD to the specified state
   rm         Remove files from the working tree and from the index

examine the history and state (see also: git help revisions)
   bisect     Use binary search to find the commit that introduced a bug
   grep       Print lines matching a pattern
   log        Show commit logs
   show       Show various types of objects
   status     Show the working tree status

grow, mark and tweak your common history
   branch     List, create, or delete branches
   checkout   Switch branches or restore working tree files
   commit     Record changes to the repository
   diff       Show changes between commits, commit and working tree, etc
   merge      Join two or more development histories together
   rebase     Reapply commits on top of another base tip
   tag        Create, list, delete or verify a tag object signed with GPG

collaborate (see also: git help workflows)
   fetch      Download objects and refs from another repository
   pull       Fetch from and integrate with another repository or a local branch
   push       Update remote refs along with associated objects

'git help -a' and 'git help -g' list available subcommands and some
concept guides. See 'git help <command>' or 'git help <concept>'
to read about a specific subcommand or concept.

\end{lstlisting}
